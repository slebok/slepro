\documentclass[preprint,authoryear,12pt]{noelsarticle}
%\documentclass[preprint,12pt]{elsarticle}
%\documentclass[final,3p,times,authoryear]{elsarticle}
%\documentclass[final,3p,times]{elsarticle}

\usepackage{tikz}
\usepackage{hyperref}
\usepackage{verbatim}
\usepackage{comment}
\usepackage{boxedminipage}
\usepackage{amssymb}
\usepackage{listings,lsthaskell,lstprolog,lstegl,lstesl,lstfsml,lstjava,lstdot,lsttext,lstueber}

\newcommand{\m}[1]{\ensuremath{\mathit{#1}}}
\newcommand{\slepro}{\textsc{SlePro}}
\newcommand{\concept}[1]{\emph{#1}}
\definecolor{codebackground}{rgb}{0.97,0.97,0.97}
\definecolor{gray}{rgb}{0.5,0.5,0.5}
\definecolor{keyword}{rgb}{0.5,0.0,0.5}
\definecolor{comment}{rgb}{0.3,0.5,0.3}
\definecolor{code}{rgb}{0.0,0.0,0.3}
\definecolor{ncode}{rgb}{0.0,0.3,0.3}
\definecolor{scode}{rgb}{0.3,0.0,0.3}
\definecolor{static}{rgb}{0.0,0.0,0.3}
\newcommand{\ncode}[1]{\texttt{\textup{\color{ncode}#1}}}
\newcommand{\jcode}[1]{\texttt{\textup{\color{code}#1}}}
\newcommand{\scode}[1]{\texttt{\textup{\color{scode}#1}}}
\newcommand{\static}[1]{\texttt{\textsl{\color{static}#1}}}

\newcommand{\codefigure}[3]{
\begin{figure}[t!]
\begin{boxedminipage}{\hsize}
\mbox{}\hfill{}{\small\textit{\href{http://github.com/slebok/slepro/tree/master/#2}{#2}}}
\lstinputlisting[language=#3]{../../#2}
\end{boxedminipage}
\caption{#1.}
\label{#2}
\medskip
\end{figure}}

\lstset{%
  fontadjust=true,%
  showstringspaces=false,%
  basicstyle=\small\sffamily,%
  keywordstyle=\color{keyword},%
  commentstyle=\color{comment}\rmfamily\itshape\small,%
  columns=fullflexible,%
  tabsize=2,
%  numbers=left,                   % where to put the line-numbers
  numberstyle=\tiny\color{gray},  % the style that is used for the line-numbers
  backgroundcolor=\color{codebackground},
  stepnumber=2,                   % the step between two line-numbers
%  numbersep=5pt,                  % how far the line-numbers are from the code
 % aboveskip=1.5\medskipamount,%
 % belowskip=1.5\medskipamount,
aboveskip = 0pt,
belowskip = 0pt,
  %breaklines=true
}

\newcommand{\wikipediafn}[2]{\footnote{\url{#1}---last visited #2.}}
\newcommand{\ooo}[1]{\textsf{101#1}}
\newcommand{\ueber}{\textsf{ueber}}
\newcommand{\ppl}{\textsf{ppl}}
\newcommand{\esl}{\textsf{esl}}

%% \biboptions{comma,round}
% \biboptions{}

\begin{document}

\begin{frontmatter}

\title{The megamodeling language \ueber}

\author{Ralf L\"ammel and Andrei Varanovich}

\address{Software Languages Team\\University of Koblenz-Landau, Germany}

\begin{abstract}
  We describe the megamodeling language \ueber, as it is used for
  modeling the linguistic architecture of the Prolog-based \slepro{}
  project; see \url{https://github.com/slebok/slepro}.

\bigskip

\begin{comment}
\noindent
\textbf{Acknowledgement}: {\small This document and the underlying
  development are parts of a broader effort on language modeling and
  software language engineering---joint work with \emph{Anya Helene
    Bagge}, University of Bergen. Helpful interaction with and
  feedback by Andrei Varanovich, University of Koblenz-Landau, is also
  gratefully acknowledged.}
\end{comment}

\medskip

\noindent
\textbf{Version}

0.00001 as of 26 November 2013.

\medskip

\noindent
\textbf{Repository location of document}: 

\url{https://github.com/slebok/slepro/tree/master/docs/ueber}.
\end{abstract}

\end{frontmatter}

\pagebreak

%%%%%%%%%%%%%%%%%%%%%%%%%%%%%%%%%%%%%%%%%%%%%%%%%%

\tableofcontents

\pagebreak

%%%%%%%%%%%%%%%%%%%%%%%%%%%%%%%%%%%%%%%%%%%%%%%%%%

\section{Introducing \ueber}

The notion of megamodel has been defined and used in different ways in
software engineering. We favor here the view that a megamodel models
the architecture of a software system with particular focus on the
involved languages, technologies, and general software
concepts~\citep{FavreLV12}

In this document, we present a particular megamodeling language,
\ueber, which is used for megamodeling the Prolog-based \slepro{}
project, which is essentially a repository of language modeling
artifacts: parsers, translators, semantics definitions, etc.

In the following example, we demonstrate \ueber{} in a way as if we
were testing a language processor. Megamodeling is in no way focused
or restricted to software testing, but it happens that the particular
setup of the \slepro{} repository lets megamodeling be useful for
testing.

\codefigure{%
A \ppl{} expresion composing text literals vertically}{%
languages/ppl/test/vbox.ppl}{%
prolog}

\codefigure{%
The text rendered for the \ppl{} expression of \autoref{languages/ppl/test/vbox.ppl}}{%
languages/ppl/test/vbox.txt}{%
text}

We are concerned with a simple language \ppl---the Pretty Printing
Language.\footnote{\url{https://github.com/slebok/slepro/tree/master/languages/ppl}}
This language offers a number of simple composition combinators for
composing text vertically and horizontally. Consider the \ppl{}
expression of \autoref{languages/ppl/test/vbox.ppl}. \ppl{} comes with
a simple evaluation semantics that essentially renders the \ppl{}
expressions; see \autoref{languages/ppl/test/vbox.txt}.

\codefigure{%
Evaluation rule for vertical boxes}{%
languages/ppl/eval/vbox.pro}{%
prolog}

\codefigure{%
Final rendering step: turn lists of lines into a single string}{%
languages/ppl/to-string.pro}{%
prolog}

\autoref{languages/ppl/eval/vbox.pro} shows one rule of the evaluator
for \ppl{} terms---the rule for vertical
composition. \autoref{languages/ppl/to-string.pro} shows the trivial
conversion from lists of lines to a single string. This is a
straigthforward Prolog-based implementation of the pretty printing
combinators. The details do not matter much to the megamodeling
discussion that follows.

\codefigure{%
Syntax of \ppl{} terms}{%
languages/ppl/as.esl}{%
esl}

The syntax of all \ppl{} terms is defined in
\autoref{languages/ppl/as.esl}. We use a simple language, \esl, for
term algebras or algebraic data types, which we do not discuss here in
detail. The notation should be straightforward; we declare a number of
different symbols (`constructors') for different kinds of \ppl{}
combinators. The sample term of \autoref{languages/ppl/test/vbox.ppl}
conforms to the signature.

\codefigure{%
A megamodel for testing \ueber}{%
languages/ppl/test/.ueber}{%
ueber}


%%%%%%%%%%%%%%%%%%%%%%%%%%%%%%%%%%%%%%%%%%%%%%%%%%

{\small

%\setlength{\bibsep}{4pt}
\bibliographystyle{elsarticle-harv}
%\bibliographystyle{elsarticle-num}
\bibliography{paper}

}

%%%%%%%%%%%%%%%%%%%%%%%%%%%%%%%%%%%%%%%%%%%%%%%%%%

\end{document}
